\documentclass[a4paper, notitlepage]{report}
\usepackage{graphicx}

\usepackage{subcaption}
\usepackage{hyperref}
\usepackage{algpseudocode}
\usepackage{algorithm}
\usepackage{amsmath}
\usepackage{listings}

\setlength{\parskip}{1em}

\usepackage{titlesec, color}
\usepackage[Lenny]{fncychap}


%% \usepackage[T1]{fontenc}
%% \usepackage{titlesec, blindtext, color}
%% \definecolor{gray75}{gray}{0.75}
%% \newcommand{\hsp}{\hspace{20pt}}
%% \titleformat{\chapter}[hang]{\Huge\bfseries}{\thechapter\hsp\textcolor{gray75}{|}\hsp}{0pt}{\Huge\bfseries}



\usepackage{etoolbox}

\lstset{
  breaklines=true,
  basicstyle=\ttfamily,
  numbers=left
}

\makeatletter
\patchcmd{\chapter}{\if@openright\cleardoublepage\else\clearpage\fi}{}{}{}

\title{Cryptol: A Domain Specific Language for  Verification of Cryptographic Algorithms}
\author{Zhiyuan Lin}
\date{\today}



\begin{document}

%{\let\newpage\relax\maketitle}
\maketitle
\begin{abstract}

\end{abstract}

\newpage
\vspace{30pt}



\chapter{Introduction}

\section{Organizations}

\chapter{Literature Survey}

In this chapter we present previous works that uses programming
language as a means to improve reliability of cryptographic applications.
In Section~\ref{sec:crypto_lang}, we give broad overview of several different
works, whereas in Section~\ref{sec:cryptol} we dig into the details of
the Cryptol programming language and the guarantees it provides.

\section{Languages for Cryptographic Applications}
\label{sec:crypto_lang}

\section{The Cryptol Language}
\label{sec:cryptol}

\chapter{Implementation and Empirical Evaluation}

This section presents the implementation part of the project. 

\section{Design}

\section{High Assurance Programming with Cryptol}

% The plan has been realized successfully without hiccups.
% Followed the schedules. 

The aim of this project is to investigate domain specific languages
that enhance reliability of cryptographic algorithms. The focus of this
work is on Cryptol, however other languages with similar facilities will
also be covered as related works.

\begin{figure}[h]
  \centering
\begin{tabular}{| l | p{5cm} |}
  \hline
  May 25th - June 2nd: & Surveying Cyptol and similar languages \\
  \hline
  June 2nd - June 9th: & In-depth investigation of Cryptol's design and theory \\
  \hline
  June 9th - June 23rd: & Implementing AES algorithm in Cryptol \\
  \hline
  June 23rd - July 30th: & Verification of implementaion using Cryptol and SAW. \\
  \hline
  July 30th - July 7th: & Finishing up evaluation and writing report \\
  \hline
  July 7th - July 14th: & Preparing for presentation \\
  \hline
\end{tabular}
\caption{Time Line of The Project}
\label{fig:time}
\end{figure}


Therefore the first part of the project will be a brief survey of Cryptol
and similar languages. This part will also cover high-level design of
the Cryptol language and theory behind its functions for formal verification.

The second part of the project will be empirical study of the Cryptol language
with an implementation.
For the implementation part, the AES symmetric-key algorithm
\cite{standard2001announcing, bertoni2002efficient} is to be implemented
 and properties related to the algorithm will be defined and checked
in Cryptol to evaluate whether the language can efficiently verify
the implementation. AES is chosen because it is the modern standard
for symmetric-key encryption is widely used. 
The focus in this part is the evaluation of the language,
rather than actual implementation of the algorithm, therefore other
algorithms written in Cryptol, if available, will
also be used to conduct empirical evaluations.

Another tool provided as a part of the Cryptol project is called
The Software Analysis Workbench (SAW). SAW also provides formal verification
for properties of programs written in Cryptol.
SAW utilizes symbolic execution to translate programs into formal models.
This tool will also be used to verify the AES implementation in Cryptol
in order to see if it provides better functionalities for verification.

Figure~\ref{fig:time} provides a time frame for the project. This is just
a rough estimation, but the project will follow the steps specified.
As mentioned before, because the focus is investigation
of the Cryptol language, more time will be spent on evaluating the language
functionalities. 




\newpage

\bibliographystyle{ieeetr}
\bibliography{proposal}

\end{document}
